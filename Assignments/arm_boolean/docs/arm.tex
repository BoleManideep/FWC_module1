\documentclass[10pt,twocolumn]{article}
\usepackage[latin1]{inputenc}
\usepackage{amsfonts}
\usepackage{amssymb}
\usepackage{graphicx}
\usepackage{hyperref}
\usepackage{karnaugh-map}
\usepackage[cmex10]{amsmath}
\usepackage{array}
\usepackage{booktabs}
\usepackage[margin=0.5in]{geometry}
\usepackage{tikz}
\usetikzlibrary[karnaugh]\usetikzlibrary{arrows,shapes.gates.logic.US,shapes.gates.logic.IEC,calc}

\title{\textbf{ARM Assignment}}
\author{Bole Manideep}
\date{October 2022}

\begin{document}
\maketitle
\paragraph{\textit{Problem Statement} - Draw a logic circuit for the boolean expression, (U + V').W' + Z and verify it's functionality.}

\tableofcontents

\section{Components}

\begin{center}
    \setlength{\arrayrulewidth}{0.1mm}
	\setlength{\tabcolsep}{12pt}
	\renewcommand{\arraystretch}{1.5}
    \begin{tabular}{|c|c|c|}
    \hline 
    \textbf{S.No} & \textbf{Component} & \textbf{Number}\\ \hline
	1. & Vaman Board & 1 \\
	2. & Bread Board & 1 \\
	3. & Jumer Wires(F-M) & 10 \\
	4. & LED & 1 \\
	5. & Resistor(150 ohm) & 1 \\ \hline   
   \end{tabular}
\end{center}
\section{Description}
\large
Given is a boolean expression with four different
variables implying that four inputs are to be given for the
circuit,in addition to that we have some mathematical
operations, apostrophe and dot operators. \vspace{2mm} 
\\ These symbols are nothing but the logic gates
represting AND, OR, NOT gates for symbols "\textbf{.}", "\textbf{+}", " \textbf{'} " respectively. \vspace{2mm}
\\ So, for the given expression 5 distinct logic gates are to
be connected using the four inputs in accordance with the
boolean expression to excute the logic in pratice.

\section{Logic Circuit}

\begin{center}
\tikzstyle{branch}=[fill,shape=circle,minimum size=3pt,inner sep=0pt]

\begin{tikzpicture}[label distance=2mm]
    \node (x0) at (0,0) {$U$};
    \node (x1) at (1,0) {$V$};
    \node (x2) at (2,0) {$W$};
    \node (x3) at (3,0) {$Z$};
    
    \node[not gate US, draw, rotate=-90] at ($(x1)+(0,-1)$) (Not1) {};
    \node[not gate US, draw, rotate=-90] at ($(x2)+(0,-1)$) (Not2) {};

    \node[or gate US, draw, logic gate inputs=nn] at ($(x3)+(1,-2)$) (Or1) {};

    \node[and gate US, draw, logic gate inputs=nn, anchor=input 1] at ($(Or1.output)+(0.5,-1)$) (And1) {};
    \node[or gate US, draw, logic gate inputs=nn, anchor=input 1] at ($(Or1.output -| And1.output)+(0.5,-2)$) (Or2) {};

    \foreach \i in {2,1}
    {
        \path (x\i) -- coordinate (punt\i) (x\i |- Not\i.input);
        \draw (punt\i) node[branch] {} -| (Not\i.input);
    }
    \draw (x0) |- (Or1.input 2);
    \draw (Not1.output) |- (Or1.input 1);
    \draw (Or1.output) -- ([xshift=0.3cm]Or1.output) |- (And1.input 1);
    \draw (Not2) |- (And1.input 2);
    \draw (And1.output) -- ([xshift=0.3cm]And1.output) |- (Or2.input 1);
    \draw (x3) |- (Or2.input 2);
    \draw (Or2.output) -- ([xshift=0.5cm]Or2.output) node[above] {$F$};   

\end{tikzpicture} \vspace{4mm}
\\ \textbf{Logic Circuit for "F = (U + V').W' + Z" }
\end{center}


\section{Truth Table}

\begin{center}
    \setlength{\arrayrulewidth}{0.5mm}\setlength{\tabcolsep}{18pt}
	\renewcommand{\arraystretch}{1.5}
    \begin{tabular}{|c|c|c|c|c|}
    \hline 
    \textbf{$U$} & \textbf{$V$} & \textbf{$W$} & \textbf{$Z$} & \textbf{$F$}		\\ \hline
    0 & 0 & 0 & 0 & 1\\
    0 & 0 & 0 & 1 & 1\\
    0 & 0 & 1 & 0 & 0\\
    0 & 0 & 1 & 1 & 1\\
    0 & 1 & 0 & 0 & 0\\
    0 & 1 & 0 & 1 & 1\\
    0 & 1 & 1 & 0 & 0\\
    0 & 1 & 1 & 1 & 1\\
    1 & 0 & 0 & 0 & 1\\
    1 & 0 & 0 & 1 & 1\\
    1 & 0 & 1 & 0 & 0\\
    1 & 0 & 1 & 1 & 1\\
    1 & 1 & 0 & 0 & 1\\
    1 & 1 & 0 & 1 & 1\\
    1 & 1 & 1 & 0 & 0\\
    1 & 1 & 1 & 1 & 1\\ \hline
   	\end{tabular}
\end{center}

\section{Procedure}
\raggedright 1.After executing the following code using make, a binary file is generated with .bin extension in the output directory. \vspace{2mm} \\ 
\centering \underline{\href{https://github.com/BoleManideep/FWC_module1/tree/main/Assignments/assembly/codes}{"Code"}} \\ \vspace{2mm} 
\raggedright 2.Now from the termux, using scp protocol, send the generated bin file to the laptop. \\ \vspace{2mm}
\raggedright 3.There we are supposed to flash the .bin file into the ARM through the terminal.\\ \vspace{2mm}
\raggedright 4.After flashing, reset the Vaman board.\\ \vspace{2mm}
\raggedright 5.Make connections between the LED and ARM board using jumper wires. \\ \vspace{2mm}
\raggedright 6.Now check the output with reference to the truth table present above.

\section{K-Map}

From the Truth Table we can draw a K-map and it is as follows:

\begin{karnaugh-map}[4][4][1][$WZ$][$UV$]
        \minterms{0,1,3,5,7,8,9,11,12,13,15}
        \maxterms{2,4,6,10,14}
        \implicant{1}{11}
        \implicant{12}{9}
        \implicantedge{0}{0}{8}{8}
    \end{karnaugh-map}

\section{Conclusion}
Hence implemented the given boolean expression and drawn it's corresponding Logic Circuit after verifying it's functionality using ARM.
\end{document}
